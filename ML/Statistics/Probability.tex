\documentclass[12pt, letterpaper]{article}
\title{Basic probability}
\author{Sergi Boquera}
\date{November 2024}
\usepackage{graphicx}
\usepackage{amsmath}

\newtheorem{definition}{Definition}
\newtheorem{theorem}{Theorem}
\newtheorem{lemma}[theorem]{Lemma}
\newtheorem{corollary}{Corollary}


\begin{document}
	\maketitle
	
	%First section of the book statistical inference by casella and berger%
	\section{Set theory}
	
		\begin{definition}
			The set, S, of all possible outcomes of a particular experiment is called the \textbf{sample space} for the experiment.
		\end{definition}
		
		\begin{definition}
			An \textbf{event} is any collection of posible outcomes of an experiment, that is, any subset of S.
		\end{definition}
		
		Let A, be an event of a subset S. We say that the event A occurs if the outcome of the experiment is in the set A.
		
	
	
	\section{Axiomatc probability theory}
	
		When an experiment is performed, the realization of the experiment is an outcome on the sample space. When a experiment is repeated a number of times, different outcomes may occur each time, or some outcomes may repeat. This "frequency of occurrence" of an outcome can be though of as a probability.\newline
		
		
		For each event A in the sample space S, we want to associate with A a number between zero and one that will be called the probability of A, denoted by P(A).
		
	
	
	
\end{document}