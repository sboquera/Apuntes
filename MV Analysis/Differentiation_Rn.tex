\documentclass[12pt, letterpaper]{article}
\title{Differentiation on $R^n$}
\author{Sergi Boquera}
\date{November 2024}
\usepackage{graphicx}
\usepackage{amsmath}
\usepackage{amsthm}

\newtheorem{definition}{Definition}[section]
\newtheorem{theorem}{Theorem}[section]
\newtheorem{lemma}[theorem]{Lemma}
\newtheorem{corollary}{Corollary}


\begin{document}
	\maketitle
	
	
	
	
	\section{Single variable derivative}
	
	\begin{definition}
		Let A be a subset of R; Let 
		\[ \phi'(a) = \lim_{t\to0} \frac{\phi(a+t) - \phi(a)}{t} \]
		
		provided the limit exists, we say that $\phi$ is differentiable at a.
	\end{definition}
	
	 The following facts are inmediat consequence:
	\begin{itemize}
		\item Differentiable functions are continuous.
		\item Composites of differentiable functions are differentiable.
	\end{itemize}
	We seek to define the derivativo of a function $f$ mapping a subset of $R^m$ into $R^n$ which preserves the properties that we have previously mentioned.\newline
	
	
	
	
	\section{Multivariable differentiation}
	
		\begin{definition}
			Let $A \in R^m$; lef $f: A \to R^n $. Suppose A contains a neighbourhood of a. Given $u \in R^m$ with $u \ne 0$, define
				\[f'(a;u) = \lim_{t \to 0}\frac{f(a+tu)-f(a)}{t}\]
			provided the limit exists. This limit is called \textbf{directional derivative} of $f$ at $a$ with respecto to the vector $u$.\newline
		\end{definition}
		
	
		\begin{definition}
			Let $A \in R^m$, let $f: A \to R^n$. Suppose A contains a nighborhood of a. We say that $f$ is \textbf{differentiable at a} if there is a n by m matrix B such that
				\[ \lim_{h \to 0} \frac{f(a+h) - f(a) - Bh}{||h||} = 0\]
			The matrix B, whisch is unique, is called the derivative of f at a; it is denoted $Df(a)$.\newline
		\end{definition}
		
		Next, we will prove the following facts about differentiable functions:
		\begin{itemize}
			\item Differentiable functions are continuous.
			\item Composites of differentiable functions are differentiable.
			\item Differentiablilty of $f$ at a implies the existence of all the directional derivatives of f at a.\newline
		\end{itemize}
		
		
		\begin{theorem}
			Let $A \in R^m$; let $f: A \to R^n$. If f is differentiable at a, then all directional derivatives of $f$ at a exist, and
				\[f'(a;u) = Df(a) \cdot u\].
		\end{theorem}
		\begin{proof}
			Let $B = Df(a)$. Set $h = tu$ in the definition of differentiability, where $t \ne 0$. Then by hypothesis,
			
				
				\[^* \lim_{t \to 0} \frac{f(a+tu) - f(a) - Btu}{||tu||} = \lim_{t \to 0} \frac{f(a+tu) - f(a)  - tBu}{|t|\cdot||u||} = 0\]
				
			If $t \to 0$ through positive values, we multiply * by $||u||$ to conclude that
				\[ \lim_{t \to 0} \frac{f(a+tu) - f(a)}{t} - Bu = 0\]
			as $t \to 0$ as desired.
			If t approaches 0 through negative values, we multiply (*) by $-||u||$ to reach the same conclusion.\newline
		\end{proof}		
		
		
		\begin{theorem}
			Let $A \in R^m$; let $f:A \to R^n$. If $f$ is differentiable at a, then $f$ is continuous at a.
		\end{theorem}
		\begin{proof}
			Let $B = Df(a)$. For h near 0, write\newline 
				\[ f(a+h) - f(a) = ||h||\cdot\left[ \frac{f(a+h) - f(a) - B\cdot||h||}{||h||}\right] - B||h||\]
			By hypothesis, the expression in brackets approaches 0 as h approaches 0.
			Then 
				\[\lim_{h\to0}\left[f(a+h) - f(a)\right] = 0\]\newline
		\end{proof}
		
		
		
		We now center our attention to the concept of partial derivatives. Which will be of great use calculating derivatives of aribitrary order.\newline
		
		\begin{definition}
			Let $A \in R^m$; let $f: A \to R$. We define the $j^{th}$ \textbf{partial derivative} of $f$ at a to be the directional derivative of $f$ at a with respect to the vector $e_{j}$. Provided this derivative exists we denote it by $D_{j}f(a)$. \newline
		\end{definition}
		
		
		
		\begin{theorem}
			Let $A \in R^m$; let $f: A \to R$. If $f$ is differentiable at a, then
				\[Df(a) = \left[D_{1}f(a), D_{2}f(a), ... , D_{m}f(a)\right]\]
		\end{theorem}
		\begin{proof}
		By hypothesis, $Df(a)$ exists and is a matrix of size 1 by m. Let
			\[Df(a) = \left[\lambda_{1}, \lambda_{2}, ..., \lambda_{m}\right]\]
		It follows (Theorem 2.1) that
			\[D_{j}f(a) = f'(a;e_{j}) = Df(a)\cdot e_{j} = \lambda_{j}\]
		\end{proof}
		
		
		
		
		
		
		
		
		
		
		
		
		
		
		
		
		
		
		
		
		
		
		
		
		
		
		
		
		
		
		
		
\end{document}